\documentclass[9pt]{beamer}
\usepackage[utf8]{inputenc}
\usepackage[T1]{fontenc}
\usepackage[french]{babel}
\usepackage{graphicx}
\usepackage{amsmath}
\usepackage{amssymb}
\usepackage{hyperref}

\title{Présentation du projets de programmation impérative}
\author{\textsc{Breuil} Dorian, \textsc{Defay} Adrien et \textsc{Peyrard} Gaultier}
\date{\today}

\begin{document}

\frame{\titlepage}  % Crée la première page avec le titre

\begin{frame}
  \frametitle{Table des matières}
  \tableofcontents
  
\end{frame}

\section{Introduction}
\begin{frame}
  \frametitle{Introdution}  % Titre de la diapositive
  % Contenu de la diapositive
  Voici nos choix de pensées pour notre projet.
\end{frame}

\section{Module et structure du code}
\begin{frame}
  \begin{center}
      \Large \textbf{Module et structure du code}
  \end{center}
\end{frame}

\begin{frame}
  \subsection{Organisation du code}
  \frametitle{Organisation du code}
  Le code est organisé en plusieurs modules :
  \begin{itemize}
  \item \textbf{Main} : module principal
  \item \textbf{Gamewindow} : gestion de la fenêtre de jeu
  \item \textbf{Plateau} : gestion du plateau de jeu
  \item \textbf{Arbre} : gestion de l’arbre de décisions
  \item \textbf{Coordonnee} : gestion des coordonnées
  \end{itemize}
\end{frame}

\begin{frame}
  \section{Types abstrait de données}
  \frametitle{Organisation du code}
\end{frame}

\subsection{Structure plateau}
\begin{frame}
  \frametitle{Structure plateau}
  \begin{itemize}
    \item \textit{mat} : tableau 2D (0 = vide, 1 = noir, 2 = blanc, 4 = jouable)
    \item \textit{l}, textit{c} : dimensions du plateau
    \item \textit{joueur}, \textit{bot} : couleur respective (1 = noir, 2 = blanc)
    \item \textit{scoreb}, \textit{scoren} : scores des Blancs et Noirs
  \end{itemize}
\end{frame}

\subsection{Structure arbre}
\begin{frame}
  \frametitle{Structure arbre}
  \begin{itemize}
    \item \textit{val} :  valeur du coup (basée sur une grille stratégique)
    \item \textit{coord} : coordonnées du coup
    \item \textit{branches} : tableau de sous-arbres (coups suivants)
    \item \textit{nb\_fils} : nombre de coups enfants
  \end{itemize}
\end{frame}

\section{Structure de données}
\begin{frame}
  \frametitle{Structure de données}
  \subsection{coordonnées}
  \textit{coordonnee}
  \begin{itemize}
    \item \textit{x} ,\textit{y} : entiers (coordonnées)
  \end{itemize}

~

  \subsection{tab\_coordonnées}
  \textit{tab\_coordonnee}
  \begin{itemize}
    \item \textit{len} : taille du tableau
    \item \textit{tab} : tableau de coordonnées
  \end{itemize}
\end{frame}

\section{Fonctionnement de l'IA}
\begin{frame}
  \begin{center}
      \Large \textbf{Fonctionnement de l'IA}
  \end{center}
\end{frame}

\subsection{Phases 1 \& 2}
\begin{frame}
  \frametitle{Phases 1 \& 2}
  L’IA sélectionne une case aléatoirement parmi les coups valides.

  C’est la version la plus simple.
\end{frame}

\subsection{Phases 3}
\begin{frame}
  \frametitle{Phases 3}
  Utilisation d’une \textbf{matrice d’évaluation} des positions :

  \begin{tabular}{|c|c|c|c|c|c|c|c|}
    \hline
    COIN & BORD & BORD & BORD & BORD & BORD & BORD & COIN \\
    \hline
    BORD & DANG & MAUV & MAUV & MAUV & MAUV & DANG & BORD \\
    \hline
    BORD & MAUV & BASE & BASE & BASE & BASE & MAUV & BORD \\
    \hline
    BORD & MAUV & BASE & DEFA & DEFA & BASE & MAUV & BORD \\
    \hline
    BORD & MAUV & BASE & DEFA & DEFA & BASE & MAUV & BORD \\
    \hline
    BORD & MAUV & BASE & BASE & BASE & BASE & MAUV & BORD \\
    \hline
    BORD & DANG & MAUV & MAUV & MAUV & MAUV & DANG & BORD \\
    \hline
    COIN & BORD & BORD & BORD & BORD & BORD & BORD & COIN \\
    \hline
  \end{tabular}
  \textbf{Légende} :
  \begin{itemize}
    \item \textit{4 (COIN)} : coins, prioritaires car irréversibles
    \item \textit{2 (BORD)} : bords avantageux
    \item \textit{1 (BASE)} : positions sûres
    \item \textit{0 (DEFA)} : neutres
    \item \textit{-1 (MAUV)} : risquées
    \item \textit{-2 (DANG)} : à éviter absolument
  \end{itemize}
  Utilisation de \textit{simuler\_cou\_prof\_3()} :

Elle simule 2 coups du joueur et 2 coups du bot. Elle retourne un arbre d’évaluation des coups possibles.

Si plusieurs coups ont la même valeur, le choix est fait aléatoirement.
\end{frame}

\subsection{Phases 4}
\begin{frame}
  \frametitle{Phases 4}
  Approche \textbf{récursive} avec \textit{simuler\_coup\_prof\_n()} :

  Cette fonction permet de simuler une profondeur arbitraire de coups pour l’IA.

  Elle prend en argument :
  \begin{itemize}
    \item Un plateau
    \item La couleur du bot
    \item La profondeur de l’arbre à calculer
  \end{itemize}
\end{frame}

\section{Difficultées rencontrées et solution trouvées}
\begin{frame}
  \frametitle{Difficultés rencontrées et solution trouvées}
  \begin{itemize}
  \item D : Comment faire pour parcourir l'arbre sans se perdre et comment faire pour économiser de la mémoire ?
  \item S : Dans le cas de l'étape 3, 3 cas différantes dans la fonction : profondeur 0, 1 et 2.   
  \end{itemize}
\end{frame}

\section{Difficultées non résolues}
\begin{frame}
  \begin{center}
      \Large \textbf{Difficultées non résolues}
  \end{center}
\end{frame}

\section{Conclusion}
\begin{frame}
  \frametitle{Conclusion}
  Merci de votre attention
\end{frame}

\end{document}
